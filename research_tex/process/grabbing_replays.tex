\begin{document}

\section{Grabbing Replays}

\subsection{Replay Downloading}

We need to look at $replays$ as a mean to find out how well the player does (in other words, $difficulty$ of the map)

\paragraph{osu!API GET replay} The API provides us with the replay file itself. Not going into detail, we are able to extract all key taps (including releases) of the player during the play.

These files are not saved, instead they are instantly decoded into the following format.

\subsection{Replay Decoding}

The format we get from running the python code goes as follows

$$ action_{replay} := \lbrace(offset_1, action_1), (offset_2, action_2), ... , (offset_n, action_n)\rbrace $$

Whereby, 
$$n \in \lbrace-9, -8, ... , -2, -1, 1, 2, ... , 8, 9\rbrace$$

$offset$ is when the $action$ happens. For $action$, $-n$ means the key \textbf{n} is released, $n$ means the key \textbf{n} is pressed.

We will save this data in a file with $<beatmap\_id>.acr$ extension.

\subsection{Replay Validation}

As discussed in the preface, we need to resolve two issues. \textbf{Multiple Replays} and \textbf{Multiple Players}, we will talk about it in a future section, but for this, we will create a different data file with $<beatmap\_id>.acrs$ extension.

\end{document}
