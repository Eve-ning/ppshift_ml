\begin{document}
\section{Difficulty from Replays}

We are able to extract data from replays, however, they are not linked to the beatmaps themselves. This means that it only has data of what keys and when did the player press it, there's not data on what accuracy did the player.

To recap, we managed to decode the replay sent into the following format:

$$ action_{replay} := \lbrace(offset_1, action_1), (offset_2, action_2), ... , (offset_n, action_n)\rbrace $$

Whereby, 
$$n \in \lbrace-9, -8, ... , -2, -1, 1, 2, ... , 8, 9\rbrace$$

$offset$ is when the $action$ happens. For $action$, $-n$ means the key \textbf{n} is released, $n$ means the key \textbf{n} is pressed.

\subsection{Mapping a Replay to Beatmap}

In this, we match all similar actions in their respective columns.

There are a few things we need to take note of when matching:
\begin{enumerate}
	\item Not all $action_{beatmap}$ will have a matching $action_{replay}$
	\item We put the threshold of this matching as $100ms$.
	\item The nearest $action_{replay}$ will match the $action_{beatmap}$, not the earliest one.
	\item We will deviate on how osu! calculate accuracy due to the above pointers, but its difference is insignificant.
\end{enumerate}

We will expect the output of:

$$ deviation := \lbrace(offset_1, deviation_1), (offset_2, deviation_2), ..., (offset_n, deviation_n)\rbrace $$

Where:
$$ n = length(action_{beatmap}) $$

And if there's no match, $deviation = 101$, this is to allow us to understand that it's a \textbf{miss} instead of a $100ms$ hit.

However, this is a bit ugly, so what we will use is the following:
$$ accuracy := \lbrace(offset_1, accuracy_1), (offset_2, accuracy_2), ..., (offset_n, accuracy_n)\rbrace $$

Where:
$$ accuracy_n = \frac{1}{deviation_n} $$
$$ n = length(action_{beatmap}) $$

Therefore, a \textbf{miss} would simply just be $accuracy = 0$ instead of the ugly $accuracy = 1/101$

So accuracy will only span:
$$ (Miss) 0 \leq accuracy_n \leq (Perfect) 1 $$
$$ deviation_n \in [0, 1, 2, ..., 99, 100] $$

\end{document}
