\documentclass{article}
\begin{document}

\section{Define Difficulty}

	What makes a map difficult, what is a difficult map? Could it be the following?
	The map was difficult because of ...
\begin{enumerate}
	\item Failing
	\item Combo Breaks
	\item High Stamina Requirement
	\item Low Accuracy
\end{enumerate}
We discuss all of these scenarios and we will choose one to tackle, possibly integrate the other options into our calculations in the future.

\paragraph{Failing}

The most significant way that we can readily control if players fail is via \textbf{Health Drain} in which most VSRGs will implement. However, this value is inconsistent and will not provide useful information on higher \textbf{Health Drain} values due to lack of players passing certain maps.

\paragraph{Combo Breaks}

Combo Breaks analysis is another method that isn't consistent, whereby chokes can be random, creating too much noise on higher skill plays. Combo breaks mainly can only determine the \textbf{hardest} points on the map, it doesn't depict a difficulty cure.

\paragraph{High Stamina Requirements}

While stamina is a good way to look at difficulty, it can readily be derived from accuracy, which is conveniently what we'll be looking at next

\paragraph{Low Accuracy}

This is the best way to look at difficulty, because not only it gives us a figure, it tells us the story and correlation between \textbf{accuracy} and \textbf{patterning}. This will be the main focus of the document.

\subsection{Comparing Accuracy in Replays to Patterns}

Details aside, how can we describe the impact of patterns on replays, what story does the replay tell us about the pattern?

Consider this...
\begin{enumerate}
	\item $Player_1$ plays $Pattern_A$ and $Pattern_B$
	\item $Player_1$ achieves $Accuracy_A > Accuracy_B$
	\item Considering $Accuracy_A$ and $Accuracy_B$ are independent events
	\item We deduce $Pattern_A < Pattern_B$ in difficulty
\end{enumerate}

It is a simple idea to pitch, but we will have to dive into more details on how we can "teach" a machine this concept and "learn" from it!

\end{document}
